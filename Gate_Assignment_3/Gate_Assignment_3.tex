\documentclass[journal,12pt,twocolumn]{IEEEtran}

\usepackage{setspace}
\usepackage{gensymb}
\singlespacing
\usepackage[cmex10]{amsmath}

\usepackage{amsthm}

\usepackage{mathrsfs}
\usepackage{txfonts}
\usepackage{stfloats}
\usepackage{bm}
\usepackage{cite}
\usepackage{cases}
\usepackage{subfig}

\usepackage{longtable}
\usepackage{multirow}

\usepackage{enumitem}
\usepackage{mathtools}
\usepackage{steinmetz}
\usepackage{tikz}
\usepackage{circuitikz}
\usepackage{verbatim}
\usepackage{tfrupee}
\usepackage[breaklinks=true]{hyperref}
\usepackage{graphicx}
\usepackage{tkz-euclide}

\usetikzlibrary{calc,math}
\usepackage{listings}
    \usepackage{color}                                            %%
    \usepackage{array}                                            %%
    \usepackage{longtable}                                        %%
    \usepackage{calc}                                             %%
    \usepackage{multirow}                                         %%
    \usepackage{hhline}                                           %%
    \usepackage{ifthen}                                           %%
    \usepackage{lscape}     
\usepackage{multicol}
\usepackage{chngcntr}

\DeclareMathOperator*{\Res}{Res}

\renewcommand\thesection{\arabic{section}}
\renewcommand\thesubsection{\thesection.\arabic{subsection}}
\renewcommand\thesubsubsection{\thesubsection.\arabic{subsubsection}}

\renewcommand\thesectiondis{\arabic{section}}
\renewcommand\thesubsectiondis{\thesectiondis.\arabic{subsection}}
\renewcommand\thesubsubsectiondis{\thesubsectiondis.\arabic{subsubsection}}
\newtheorem{theorem}{Theorem}[section]
\newtheorem{corollary}{Corollary}[theorem]
\newtheorem{lemma}[theorem]{Lemma}
\newtheorem{definition}{Definition}[section]


\hyphenation{op-tical net-works semi-conduc-tor}
\def\inputGnumericTable{}                                 %%

\lstset{
%language=C,
frame=single, 
breaklines=true,
columns=fullflexible
}
\begin{document}

\newcommand{\BEQA}{\begin{eqnarray}}
\newcommand{\EEQA}{\end{eqnarray}}
\newcommand{\define}{\stackrel{\triangle}{=}}
\bibliographystyle{IEEEtran}
\raggedbottom
\setlength{\parindent}{0pt}
\providecommand{\mbf}{\mathbf}
\providecommand{\pr}[1]{\ensuremath{\Pr\left(#1\right)}}
\providecommand{\qfunc}[1]{\ensuremath{Q\left(#1\right)}}
\providecommand{\sbrak}[1]{\ensuremath{{}\left[#1\right]}}
\providecommand{\lsbrak}[1]{\ensuremath{{}\left[#1\right.}}
\providecommand{\rsbrak}[1]{\ensuremath{{}\left.#1\right]}}
\providecommand{\brak}[1]{\ensuremath{\left(#1\right)}}
\providecommand{\lbrak}[1]{\ensuremath{\left(#1\right.}}
\providecommand{\rbrak}[1]{\ensuremath{\left.#1\right)}}
\providecommand{\cbrak}[1]{\ensuremath{\left\{#1\right\}}}
\providecommand{\lcbrak}[1]{\ensuremath{\left\{#1\right.}}
\providecommand{\rcbrak}[1]{\ensuremath{\left.#1\right\}}}
\theoremstyle{remark}
\newtheorem{rem}{Remark}
\newcommand{\sgn}{\mathop{\mathrm{sgn}}}
\providecommand{\abs}[1]{\vert#1\vert}
\providecommand{\res}[1]{\Res\displaylimits_{#1}} 
\providecommand{\norm}[1]{\lVert#1\rVert}
%\providecommand{\norm}[1]{\lVert#1\rVert}
\providecommand{\mtx}[1]{\mathbf{#1}}
\providecommand{\mean}[1]{E[ #1 ]}
\providecommand{\fourier}{\overset{\mathcal{F}}{ \rightleftharpoons}}
%\providecommand{\hilbert}{\overset{\mathcal{H}}{ \rightleftharpoons}}
\providecommand{\system}{\overset{\mathcal{H}}{ \longleftrightarrow}}
	%\newcommand{\solution}[2]{\textbf{Solution:}{#1}}
\newcommand{\solution}{\noindent \textbf{Solution: }}
\newcommand{\cosec}{\,\text{cosec}\,}
\providecommand{\dec}[2]{\ensuremath{\overset{#1}{\underset{#2}{\gtrless}}}}
\newcommand{\myvec}[1]{\ensuremath{\begin{pmatrix}#1\end{pmatrix}}}
\newcommand{\mydet}[1]{\ensuremath{\begin{vmatrix}#1\end{vmatrix}}}
\numberwithin{equation}{subsection}
\makeatletter
\@addtoreset{figure}{problem}
\makeatother
\let\StandardTheFigure\thefigure
\let\vec\mathbf
\renewcommand{\thefigure}{\theproblem}
\def\putbox#1#2#3{\makebox[0in][l]{\makebox[#1][l]{}\raisebox{\baselineskip}[0in][0in]{\raisebox{#2}[0in][0in]{#3}}}}
     \def\rightbox#1{\makebox[0in][r]{#1}}
     \def\centbox#1{\makebox[0in]{#1}}
     \def\topbox#1{\raisebox{-\baselineskip}[0in][0in]{#1}}
     \def\midbox#1{\raisebox{-0.5\baselineskip}[0in][0in]{#1}}
\vspace{3cm}
\title{ EE3900 : Gate Assignment-3}
\author{Nelakuditi Rahul Naga - AI20BTECH11029}
\maketitle
\newpage
\bigskip
\renewcommand{\thefigure}{\theenumi}
\renewcommand{\thetable}{\theenumi}
Download all latex-tikz codes from 
%
\begin{lstlisting}
https://github.com/Rahul27n/EE3900/blob/main/Gate_Assignment_3/Gate_Assignment_3.tex
\end{lstlisting}
\vspace{0.5cm}
\section{QUESTION: GATE EC 2005 Q.85}
A non-zero sequence $x(n)$ is given by:
\begin{figure}[!ht]
    \centering
    \includegraphics[width=\columnwidth] {Gate_Assignment_3_Fig_1.png}
    \caption{$x(n)$}
    \label{x(n)}
\end{figure}
\\The sequence
\begin{align}
y(n)=  
\begin{cases}
x\Bigg(\dfrac{n}{2}-1\Bigg), & \text{for } n \text{ even}\\
0, & \text{for } n \text{ odd}\nonumber
\end{cases}
\end{align}
is given by:
\begin{figure}[!ht]
    \centering
    \includegraphics[width=0.8\columnwidth] {Gate_Assignment_3_Fig_2.png}
    \caption{Option (a)}
    \label{Option (a)}
\end{figure}
\begin{figure}[!ht]
    \centering
    \includegraphics[width=0.8\columnwidth] {Gate_Assignment_3_Fig_3.png}
    \caption{Option (b)}
    \label{Option (b)}
\end{figure}
\begin{figure}[!ht]
    \centering
    \includegraphics[width=0.8\columnwidth] {Gate_Assignment_3_Fig_4.png}
    \caption{Option (c)}
    \label{Option (c)}
\end{figure}
\begin{figure}[!ht]
    \centering
    \includegraphics[width=0.8\columnwidth] {Gate_Assignment_3_Fig_5.png}
    \caption{Option (d)}
    \label{Option (d)}
\end{figure}
\section{SOLUTION}
We can write $x(n)$ as follows:
\begin{align}
x(n) &= \sum_{k=-2}^{2}\Bigg(\frac{1}{2}\Bigg)^{\abs{k}-1}\delta[n-k]\label{eq:1}
\end{align}
where $\delta[n-k]$ is the discrete unit sample function defined as follows:
\begin{align}
\delta[n-k]=\begin{cases}
1 \text{ if } n=k\\
0 \text{ otherwise}
\end{cases}
\end{align}
From \eqref{eq:1} we have for even $n$ :
\begin{align}
y(n) &= x\Bigg(\dfrac{n}{2}-1\Bigg)\\
&= \sum_{k=-2}^{2}\Bigg(\frac{1}{2}\Bigg)^{\abs{k}-1}\delta\Bigg[\frac{n}{2}-1-k\Bigg]\\
&= \sum_{k=-2}^{2}\Bigg(\frac{1}{2}\Bigg)^{\abs{k}-1}\delta[n-2(k+1)]\label{eq:2}
\end{align}
We clearly have $y(n)$ = 0 for odd $n$ from \eqref{eq:2}. The plots of $x(n)$ and $y(n)$ are given by:
\begin{figure}[!ht]
    \centering
    \includegraphics[width=\columnwidth] {Gate_Assignment_3_Fig_6.png}
    \caption{$x(n)$ and $y(n)$ vs $n$}
    \label{x(n) and y(n)}
\end{figure}

Hence the correct answer is option (a).
\end{document}
