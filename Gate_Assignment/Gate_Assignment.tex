\documentclass[journal,12pt,twocolumn]{IEEEtran}

\usepackage{setspace}
\usepackage{gensymb}
\singlespacing
\usepackage[cmex10]{amsmath}

\usepackage{amsthm}

\usepackage{mathrsfs}
\usepackage{txfonts}
\usepackage{stfloats}
\usepackage{bm}
\usepackage{cite}
\usepackage{cases}
\usepackage{subfig}

\usepackage{longtable}
\usepackage{multirow}

\usepackage{enumitem}
\usepackage{mathtools}
\usepackage{steinmetz}
\usepackage{tikz}
\usepackage{circuitikz}
\usepackage{verbatim}
\usepackage{tfrupee}
\usepackage[breaklinks=true]{hyperref}
\usepackage{graphicx}
\usepackage{tkz-euclide}

\usetikzlibrary{calc,math}
\usepackage{listings}
    \usepackage{color}                                            %%
    \usepackage{array}                                            %%
    \usepackage{longtable}                                        %%
    \usepackage{calc}                                             %%
    \usepackage{multirow}                                         %%
    \usepackage{hhline}                                           %%
    \usepackage{ifthen}                                           %%
    \usepackage{lscape}     
\usepackage{multicol}
\usepackage{chngcntr}

\DeclareMathOperator*{\Res}{Res}

\renewcommand\thesection{\arabic{section}}
\renewcommand\thesubsection{\thesection.\arabic{subsection}}
\renewcommand\thesubsubsection{\thesubsection.\arabic{subsubsection}}

\renewcommand\thesectiondis{\arabic{section}}
\renewcommand\thesubsectiondis{\thesectiondis.\arabic{subsection}}
\renewcommand\thesubsubsectiondis{\thesubsectiondis.\arabic{subsubsection}}


\hyphenation{op-tical net-works semi-conduc-tor}
\def\inputGnumericTable{}                                 %%

\lstset{
%language=C,
frame=single, 
breaklines=true,
columns=fullflexible
}
\begin{document}

\newcommand{\BEQA}{\begin{eqnarray}}
\newcommand{\EEQA}{\end{eqnarray}}
\newcommand{\define}{\stackrel{\triangle}{=}}
\bibliographystyle{IEEEtran}
\raggedbottom
\setlength{\parindent}{0pt}
\providecommand{\mbf}{\mathbf}
\providecommand{\pr}[1]{\ensuremath{\Pr\left(#1\right)}}
\providecommand{\qfunc}[1]{\ensuremath{Q\left(#1\right)}}
\providecommand{\sbrak}[1]{\ensuremath{{}\left[#1\right]}}
\providecommand{\lsbrak}[1]{\ensuremath{{}\left[#1\right.}}
\providecommand{\rsbrak}[1]{\ensuremath{{}\left.#1\right]}}
\providecommand{\brak}[1]{\ensuremath{\left(#1\right)}}
\providecommand{\lbrak}[1]{\ensuremath{\left(#1\right.}}
\providecommand{\rbrak}[1]{\ensuremath{\left.#1\right)}}
\providecommand{\cbrak}[1]{\ensuremath{\left\{#1\right\}}}
\providecommand{\lcbrak}[1]{\ensuremath{\left\{#1\right.}}
\providecommand{\rcbrak}[1]{\ensuremath{\left.#1\right\}}}
\theoremstyle{remark}
\newtheorem{rem}{Remark}
\newcommand{\sgn}{\mathop{\mathrm{sgn}}}
\providecommand{\abs}[1]{\vert#1\vert}
\providecommand{\res}[1]{\Res\displaylimits_{#1}} 
\providecommand{\norm}[1]{\lVert#1\rVert}
%\providecommand{\norm}[1]{\lVert#1\rVert}
\providecommand{\mtx}[1]{\mathbf{#1}}
\providecommand{\mean}[1]{E[ #1 ]}
\providecommand{\fourier}{\overset{\mathcal{F}}{ \rightleftharpoons}}
%\providecommand{\hilbert}{\overset{\mathcal{H}}{ \rightleftharpoons}}
\providecommand{\system}{\overset{\mathcal{H}}{ \longleftrightarrow}}
	%\newcommand{\solution}[2]{\textbf{Solution:}{#1}}
\newcommand{\solution}{\noindent \textbf{Solution: }}
\newcommand{\cosec}{\,\text{cosec}\,}
\providecommand{\dec}[2]{\ensuremath{\overset{#1}{\underset{#2}{\gtrless}}}}
\newcommand{\myvec}[1]{\ensuremath{\begin{pmatrix}#1\end{pmatrix}}}
\newcommand{\mydet}[1]{\ensuremath{\begin{vmatrix}#1\end{vmatrix}}}
\numberwithin{equation}{subsection}
\makeatletter
\@addtoreset{figure}{problem}
\makeatother
\let\StandardTheFigure\thefigure
\let\vec\mathbf
\renewcommand{\thefigure}{\theproblem}
\def\putbox#1#2#3{\makebox[0in][l]{\makebox[#1][l]{}\raisebox{\baselineskip}[0in][0in]{\raisebox{#2}[0in][0in]{#3}}}}
     \def\rightbox#1{\makebox[0in][r]{#1}}
     \def\centbox#1{\makebox[0in]{#1}}
     \def\topbox#1{\raisebox{-\baselineskip}[0in][0in]{#1}}
     \def\midbox#1{\raisebox{-0.5\baselineskip}[0in][0in]{#1}}
\vspace{3cm}
\title{ EE3900 : Gate-Assignment}
\author{Nelakuditi Rahul Naga - AI20BTECH11029}
\maketitle
\newpage
\bigskip
\renewcommand{\thefigure}{\theenumi}
\renewcommand{\thetable}{\theenumi}

Download all python codes from 
\begin{lstlisting}
https://github.com/Rahul27n/EE3900/blob/main/Gate_Assignment/Gate_Assignment.py
\end{lstlisting}
%
and latex-tikz codes from 
%
\begin{lstlisting}
https://github.com/Rahul27n/EE3900/blob/main/Gate_Assignment/Gate_Assignment.tex
\end{lstlisting}

\vspace{0.5cm}
\section{QUESTION: Q.33 EC-GATE-2016}

The Discrete Fourier Transform (DFT) of the 4 point sequence $ x[n] =\{3, 2, 3, 4\}$ is given by $ X[k] =\{12, 2j, 0, -2j\}$. If $X_{1}[k]$ is the DFT of the 12 point sequence $ x_{1}[n] =\{3, 0, 0, 2, 0, 0, 3, 0, 0, 4, 0, 0\}$ , the value of $\big|{\frac{X_{1}[8]}{X_{1}[11]}}\big|$ is :
\section{SOLUTION}
\vspace{0.5cm}
We can clearly observe that the sequence $x_{1}[n]$ is a interpolation in time domain of the point sequence $x[n]$ i.e;
\begin{align}
x_{1}[n] = x\Big[\frac{n}{3}\Big]
\end{align}
We know that interpolation in time domain is equivalent to replication in the frequency domain. %Therefore,
\begin{align}
 X_{1}[k] &=\{12, 2j, 0, -2j,12, 2j, 0, -2j,12, 2j, 0, -2j\}\label{eq:1}
\end{align}

Hence from \eqref{eq:1} we have:
\begin{align}
X_{1}[8] &= 12\label{eq:2}
\\ X_{1}[11] &= -2j\label{eq:3}
\end{align}
Therefore from \eqref{eq:2} and \eqref{eq:3} we have:
\begin{align}
\Big|{\frac{X_{1}[8]}{X_{1}[11]}}\Big| = \Big|{\frac{12}{-2j}}\Big| =\big|6j\big| = 6
\end{align}

\begin{figure}[!ht]
    \centering
    \includegraphics[width=\columnwidth] {GATE_Fig_1.png}
    \caption{Magnitude of $X[k]$ vs $k$}
    \label{Magnitude of X[k]}
\end{figure}

\begin{figure}[!ht]
    \centering
    \includegraphics[width=\columnwidth] {GATE_Fig_2.png}
    \caption{Magnitude of $X_{1}[k]$ vs $k$}
    \label{Magnitude of X1[k]}
\end{figure}

\end{document}

